\documentclass{article}

\usepackage{tabularx}
\usepackage{booktabs}
\usepackage[a4paper, total={6in, 8in}]{geometry}

\setlength\parindent{24pt}

\title{Problem Statement and Goals\\\progname}

\author{\authname Muhammad Ali, Winnie Trandinh, Muhammad Khan, Fady Zekry Hanna}

\date{}

%%% Comments

\usepackage{color}

\newif\ifcomments\commentstrue %displays comments
%\newif\ifcomments\commentsfalse %so that comments do not display

\ifcomments
\newcommand{\authornote}[3]{\textcolor{#1}{[#3 ---#2]}}
\newcommand{\todo}[1]{\textcolor{red}{[TODO: #1]}}
\else
\newcommand{\authornote}[3]{}
\newcommand{\todo}[1]{}
\fi

\newcommand{\wss}[1]{\authornote{blue}{SS}{#1}} 
\newcommand{\plt}[1]{\authornote{magenta}{TPLT}{#1}} %For explanation of the template
\newcommand{\an}[1]{\authornote{cyan}{Author}{#1}}

%%% Common Parts

\newcommand{\progname}{Mechatronics Engineering} % PUT YOUR PROGRAM NAME HERE
\newcommand{\authname}{Team \# 34, ParkingLotHawk
\\ Fady Zekry Hanna, zekryhf
\\ Winnie Trandinh, trandint
\\ Muhammad Ali, alim102
\\ Muhammad Khan, khanm120} % AUTHOR NAMES                  

\usepackage{hyperref}
    \hypersetup{colorlinks=true, linkcolor=blue, citecolor=blue, filecolor=blue,
                urlcolor=blue, unicode=false}
    \urlstyle{same}
                                


\begin{document}

\maketitle

\begin{table}[hp]
\caption{Revision History} \label{TblRevisionHistory}
\begin{tabularx}{\textwidth}{llX}
\toprule
\textbf{Date} & \textbf{Developer(s)} & \textbf{Change}\\
\midrule
24-09-2022 & Ali and Zaid & Initial Revision\\
26-09-2022 & Winnie and Fady & Edits \\
... & ... & ...\\
\bottomrule
\end{tabularx}
\end{table}

\section{Problem Statement}
\subsection{Problem Description}

\indent The struggle of finding a parking spot in a busy parking lot is an issue that many visitors and organizers face. The hassle of finding a parking spot results in a loss of time for the visitors, a wastage of vehicle energy, and causes frustration for visitors. All of these issues result in lost revenue and/or productivity for all stakeholders, visitors and organizers. Parking lot sizes often cannot be increased without planning and investment. Given that the number of parking spots is often a constraint, utilizing the existing parking spaces efficiently is critical to minimizing the negative impacts of this issue.


\indent Existing solutions sometimes involve a designated person, a parking lot officer, who helps to communicate parking spot availability to visitors. Of course, the primary drawback is that the operator being a person often cannot see or monitor large swaths of the parking lot accurately. Other solutions involve installing vehicle counters in the parking lot. These vehicle counters are installed either at the entrance/exits or throughout the parking lot, such as a distributed system of cameras.  These solutions are expensive and involve installations at precise locations, and thus require greater investment for organizers. Vehicle counter technology thus does not fulfill the need for many outdoor, seasonal, temporary, and/or cheaper organizers.  Examples include provincial/national parks, concerts, and religious places of worship where the highest demand for this solution is only for a certain period of time. 
\subsection{Project Description}
