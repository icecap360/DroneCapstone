\documentclass{article}
\usepackage[utf8]{inputenc}
\usepackage{fullpage}
\usepackage{amsmath, mathtools}
\usepackage{amsfonts}
\usepackage{amssymb}
\usepackage{graphicx}
\usepackage{colortbl}
\usepackage{xr}
\usepackage{hyperref}
\usepackage{longtable}
\usepackage{xfrac}
\usepackage{tabularx}
\usepackage{float}
\usepackage{siunitx}
\usepackage{booktabs}
\usepackage{caption}
\usepackage{pdflscape}
\usepackage{afterpage}


% For easy change of table widths
\newcommand{\colZwidth}{1.0\textwidth}
\newcommand{\colAwidth}{0.13\textwidth}
\newcommand{\colBwidth}{0.82\textwidth}
\newcommand{\colCwidth}{0.1\textwidth}
\newcommand{\colDwidth}{0.05\textwidth}
\newcommand{\colEwidth}{0.8\textwidth}
\newcommand{\colFwidth}{0.17\textwidth}
\newcommand{\colGwidth}{0.5\textwidth}
\newcommand{\colHwidth}{0.28\textwidth}


\newcommand{\deftheory}[9][Not Applicable]
{
\newpage
\noindent \rule{\textwidth}{0.5mm}

\paragraph{RefName: } \textbf{#2} \phantomsection 
\label{#2}

\paragraph{Label:} #3

\noindent \rule{\textwidth}{0.5mm}

\paragraph{Equation:}

#4

\paragraph{Description:}

#5

\paragraph{Notes:}

#6

\paragraph{Source:}

#7

\paragraph{Ref.\ By:}

#8

\paragraph{Preconditions for \hyperref[#2]{#2}:}
\label{#2_precond}

#9

\paragraph{Derivation for \hyperref[#2]{#2}:}
\label{#2_deriv}

#1

\noindent \rule{\textwidth}{0.5mm}

}

\begin{document}

\title{
Software Requirements Specification for ParkingLotDrone \\
  \large MECHTRON 4TB6 Capstone Design Project 
    }

\author{\authname Team \#34 \\
Fady Zekry Hanna, zekryhf \\
Winnie Trandinh, trandint \\
Muhammad Ali, alim102 \\
Muhammad Khan, khanm120}

\date{\today}
	
\maketitle

~\newpage

\pagenumbering{roman}

\tableofcontents

~\newpage

\section*{Revision History}

\begin{tabularx}{\textwidth}{p{3cm}p{2cm}X}
\toprule {\bf Date} & {\bf Version} & {\bf Notes}\\
\midrule
Date 1 & 1.0 & Notes\\
Date 2 & 1.1 & Notes\\
\bottomrule
\end{tabularx}

~\newpage

\pagenumbering{arabic}

\section{Introduction}
\label{sec:Intro}
\subsection{Purpose}
The purpose of the project will be to design an aerial drone, called ParkingLotHawk, that can detect how many parking spots are available in any given parking lot. ParkingLotHawk can be operated by property personnel to allow drivers how many spots are available at a parking lot. ParkingLotHawk will over any parking lot designated by the user and will aggregate visual information and output the amount of parking spots available. Many parking lots today do not have this data and this data can be used for many reasons.  The data can be used by retailers to know how long customers stay at specific stores or location. This data can also be used for drivers to know if there are parking spots available in a specific location allowing for drivers to not waste time and resources in a full parking lot. 
\subsection{Scope}
The product specified in this SRS is about an autonomous aerial drone for helping parking lot operators understand the state of their parking lot. The product specified does not require the operator to manually control or move the drone, rather the requirement describes various autonomous flight modes. The specified drone shall support the ability to both create a path to reach is specified location, as well as create and follow a path to explore large parking lot sections autonomously. During flight, the specified drone shall transmit live information about the parking lot sections it detects. The completed product, a combination of the physical drone and any equipment/application intended to be kept by the parking lot operator to communicate with the drone, is called the ParkingLotHawk. A solution that implements the requirements will help parking lot authorities of outdoor lots quickly gain valuable information without requiring permanent solution, large monetary and time investments or complex training. 



\subsection{Definitions, Acronyms, and Abbreviations}
\subsection{References}
\subsection{Overview}
The SRS is organized to follow the IEEE 1998 template. \nameref{sec:Intro} contains the purpose of the SRS as well as the scope of the product and the problem it solves. \nameref{sec:Desc}  refines the scope of the product further. It provides more detail about the products environment, primary functions, intended users, constraints and finally assumption. \nameref{sec:Req} contains a detailed description of all requirements, organized into sections for readability. Finally the appendix contains reflection regarding new knowledge and skills the team needs to create the specified product, along with approaches to how the team will acquire the knowledge. 
\section{Overall Description}
\label{sec:Desc}
General factors that affect the product and the requirements are described in the following subsections. A high-level overview of the product functions are also described (see \nameref{subsec:ProdFunc}). 
\subsection{Product Perspective}
The system specified is an independent and stand-alone parking lot tool. It does not fit into or interface with a larger system of parking lot and security technologies the operator may have available. 

The environment consists of an outdoor parking lot and the operator's PC. The operator's PC shall be running with Windows 10 or Windows 11.

\subsection{Product Functions}
\label{subsec:ProdFunc}

\subsubsection{Idle State}
\label{Idle State}
\subsubsection{Hover State}
\label{Hover State}
\subsubsection{Manual Location Move State}
\label{Manual Location Move State}
\subsubsection{Autonomous Explore State}
\label{Autonomous Explore State}
\subsubsection{Configure State}
\label{Configure State}
\subsubsection{Off State}
\label{Off State}
\subsubsection{Land State}
\label{Land State}
\subsubsection{Desired Location Error State}
\label{Desired Location Error State}
\subsubsection{No Parking Lot Detected Error State}
\label{No Parking Lot Detected Error State}
\subsubsection{Malfunction State}
\label{Malfunction State}

\subsection{User characteristics}
The stakeholders are Dr. Spencer Smith and the MECHTRON 4TB6 teaching assistants. The users of this product are property managers, security personnel and anyone who is an active driver. All demographics mentioned would find the data of the ParkingLotHawk useful. The knowledge expected for using the ParkingLotHawk is to be able to understand how to use a computer and how to switch a device on and off. The user will need to turn the drone on and place it close to a parking lot. The user will also need to send an ON command using a computer that will be sent to the drone using radio communication. The ParkingLotHawk is made mindful of the community, therefore, no air or noise pollution will occur, and no invasion of privacy will ensue.
\subsection{Constraints}
The purpose of the system is to provide parking spots availability to property managers and/or security personnel and this information could be relayed to customers or guests of a property. As the user can come from a non-technical background, the constrain on the usability of the product should be considered. The system will be used using radio communication from a computer and turned on and off using one button. Radio communication can only work within about 2 km from one point to the other. The project constraint present is a maximum budget of \$ 750.  Canadian regulatory policy does not allow for drone flight within 1 nautical mile (about 2 km) from heliports and 3 nautical miles (5.6 km) from airports. The team is allowed to fly the drone as it less than 250 grams in public otherwise a license would be required.
\subsection{Assumptions and dependencies}
\section{Specific requirements }
\label{sec:Req}
This section of the SRS should contain all of the software requirements to a level of detail sufficient to enable designers to design a system to satisfy those requirements, and testers to test that the system satisfies those requirements.
\subsection{External Interfaces}
\subsection{Functional Requirements}

\subsubsection{General Functional Requirements}
foo
\begin{table}[!h]
\begin{center}
\caption {GEN\_001} 
\label{GEN_001}
\begin{tabular}{ | m{3cm} | m{11cm} | }
\hline
Description & The product shall be able to recognize \nameref{Clear Boundaries}. This requirement is a refinement of the \nameref{Autonomous Explore State}. \\
\hline
Rationale & This requirement ensures that the product is able to implement basic autonomy, such as not traveling past the parking lot boundaries. \\
\hline
Phase & II \\
\hline
Likely to Change & No. This requirement is required to implement the \nameref{Autonomous Explore State}. \\
\hline
Associated Inputs and Outputs & N/A \\
\hline
\end{tabular}
\end{center}
\end{table}

\begin{table}[!h]
\begin{center}
\caption {GEN\_002} 
\label{GEN_002}
\begin{tabular}{ | m{3cm} | m{11cm} | }
\hline
Description & The product shall provide live update of c\_CurrentLoc, c\_CurrentView and c\_OccupancyMap during all normal and non-configurational operation states. This requirement is a refinement of the normal and non-configuration operation states specified in Section \ref{subsec:ProdFunc}. \\
\hline
Rationale & This requirement ensures that the product always provides the latest controlled variable information to the operator. \\
\hline
Phase & II \\
\hline
Likely to Change & No. This requirement is a part of the MVP and must be present to make the product achieve its product functions. \\
\hline
Associated Inputs and Outputs & c\_CurrentLoc, c\_CurrentView, and c\_OccupancyMap. \\
\hline
\end{tabular}
\end{center}
\end{table}

\begin{table}[!h]
\begin{center}
\caption {GEN\_003} 
\label{GEN_003}
\begin{tabular}{ | m{3cm} | m{11cm} | }
\hline
Description & The product shall allow the operator to configure the i\_MinHoverHeight, i\_MaxHoverHeight, and i\_DesiredHoverHeight. This requirement is a refinement of the \nameref{Configure State}. \\
\hline
Rationale & The value of these parameters depends on the operators view preferences and parking lot conditions. For example a parking lot with a lot of large trucks may be better suited to higher hovering hieghts. \\
\hline
Phase & I \\
\hline
Likely to Change & No. This requirement is vital to the operation of the product, as it must be suited to different parking lot environments. \\
\hline
Associated Inputs and Outputs & i\_MinHoverHeight, i\_MaxHoverHeight, and i\_DesiredHoverHeight. \\
\hline
\end{tabular}
\end{center}
\end{table}

\begin{table}[!h]
\begin{center}
\caption {GEN\_004} 
\label{GEN_004}
\begin{tabular}{ | m{3cm} | m{11cm} | }
\hline
Description & The condition i\_MinHoverHeight <= i\_DesiredHoverHeight <= i\_MaxHoverHeight shall always be true. This requirement is a refinement of the \nameref{Configure State}. \\
\hline
Rationale & This requirement ensures logical values for the parameters are set by the operator. \\
\hline
Phase & I \\
\hline
Likely to Change & No. This requirement is required to check the inputted values by the operator. \\
\hline
Associated Inputs and Outputs & i\_MinHoverHeight, i\_DesiredHoverHeight, and i\_MaxHoverHeight. \\
\hline
\end{tabular}
\end{center}
\end{table}

\begin{table}[!h]
\begin{center}
\caption {GEN\_005} 
\label{GEN_005}
\begin{tabular}{ | m{3cm} | m{11cm} | }
\hline
Description & The product shall be able to identify non-occupied parking spots. This requirement is a refinement of the the normal and non-configuration operation states specified in Section \ref{subsec:ProdFunc}. \\
\hline
Rationale & This requirement ensures that the product is able to create the occupancy map. \\
\hline
Phase & III \\
\hline
Likely to Change & No. This requirement is required to create the occupancy map, which is one of the main functions of the  \\
\hline
Associated Inputs and Outputs & i\_MinHoverHeight, i\_DesiredHoverHeight, and i\_MaxHoverHeight. \\
\hline
\end{tabular}
\end{center}
\end{table}

\subsubsection{State Implementation Requirements}
\subsubsection{State Transition Requirements}

\subsection{Performance Requirements}
foo
\begin{table}[!h]
\begin{center}
\caption {PERF\_001} 
\label{PERF_001}
\begin{tabular}{ | m{3cm} | m{11cm} | }
\hline
Description & ... \\
\hline
Rationale & ... \\
\hline
Phase & ... \\
\hline
Likely to Change & ... \\
\hline
Associated Inputs and Outputs & ... \\
\hline
\end{tabular}
\end{center}
\end{table}

foo \nameref{PERF_001}

\subsection{Logical Database Requirements}
There is no requirement for databases, as the problem requires no long-term storage of data.
\subsection{Design constraints }
refine the 750 budget
\subsection{Standards Compliance }
refine Weight+Radio
\subsection{Software System Attributes}
\subsection{Reliability }
\subsection{Availability}
\subsection{Security}
\subsection{Maintainability}
\subsection{Safety}
\subsection{Usability}
\subsection{ Portability}

\section{Supporting information}
\subsection{Appendixes}
\subsubsection{Appendix A: Reflection}
Although the team has a solid education in the foundation in Mechatronics, they lack practical experience in building drones and robots in general. The knowledge the team currently does not have is identified in the table below. Each discipline of knowledge was assigned to a specific team members who would become the expert in the subject. They may use any resources to gain the knowledge, such as books, blogs, YouTube videos, and other websites available to them. 

The team has also defined what it means to be an expert in their domain: 
\begin{itemize}
    \item In depth understanding of how the component works physically (if applicable). 
    \item Proper reasoning as to why specific component/firmware was chosen.
    \item In depth understanding of the key parameters and specifications of the product/firmware.
    \item In depth understanding of the inputs and outputs to the domain. 
    \item In depth understanding of the integration of the domain into the project. 
\end{itemize}

For many of the technical domains, the team has opted to partake in a dual-expert system where two experts are present for the domain. This ensures that in the absence of an expert, the effects on the team are minimized. Furthermore, this promotes collaboration between the experts during their research of their domain. Git, programming languages and linux are already well understood by team members. 

\begin{center}
\begin{tabular}{ | m{2cm} | m{2cm} | m{8cm} | m{2.1cm} | }
\hline
Domain & Assigned Experts & Description of Domain & Recommended Resources\\
\hline
\hline
Latex & Zaid & The Latex domain expert is responsible for understanding the Latex syntax and generation process. & Google \\
\hline
Power Management and Motors & Fady and Zaid & This domain relates to the powering of the individual components of the drone, and the control of the motors. Example components include the battery, Power Distribution Board, and the Electronic Speed Controller. & Motor datasheet, Google, Youtube \\
\hline
Mechanical Design & Winnie and Ali & Included within this domain is the creation of the custom drone frame, in addition to minimizing vibrations while maximizing structural integrity. & Google, Youtube\\
\hline
External Sensors and Peripherals & Winnie and Ali & This domain includes the external sensors that connect to the flight controller, including the camera, radio transmitter, and any other sensors that may be required. & Google, Youtube, Datasheet of components \\
\hline
Flight Controller & Zaid, Fady, Ali, and Winnie & All the members should be responsible for being experts at the Flight Controller as they all need to know how to interface with it. Further sub domains may be created at a later date if the need arises. & Official documentation of Flight Controller\\
\hline
Internal Sensors & Winnie and Ali & The Internal Sensors domain includes the sensors within the Flight Controller. Such sensors include the Inertial Measurement Unit, barometer, and GPS.  & Official documentation of Flight Controller, Google, Youtube \\
\hline
\end{tabular}
\end{center}
\subsection{Index}

\end{document}